\documentclass{beamer}
% Class options include: notes, handout, trans
%                        

% Theme for beamer presentation.
\usepackage{beamerthemesplit} 
% Other themes include: beamerthemebars, beamerthemelined,beamerthemetree, beamerthemeplain

\usefonttheme{professionalfonts}

\title[Can cream-skimming encourage prevention?]{Can cream-skimming encourage prevention?}
\subtitle{Master thesis}    % Enter your title between curly braces
\author[Hana M. Smr\u{c}kov\'{a}]{Hana Marie Smr\u{c}kov\'{a}}                 % Enter your name between curly braces
\institute[TiU]{Tilburg University}      % Enter your institute name between curly braces
\date{\today}      % Enter the date or \today between curly braces

\usepackage{graphicx}

\begin{document}

% Creates title page of slide show using above information
\begin{frame}
  \titlepage
\end{frame}
\note{Talk for 30 minutes} % Add notes to yourself that will be displayed when typeset with the notes class option.

\section[Outline]{}
% Creates table of contents slide incorporating all \section and \subsection commands.
\begin{frame}{Outline}
  \tableofcontents
\end{frame}
\section{Reserch question}
\subsection{Motivation}
\begin{frame}{Motivation}
\begin{itemize}
	\item<1-> Physical inactivity is one of the major sources of health related problems in the developed countries.
	\item<2-> Health insurance policy is likely to affect individual's incentives to be physically active.
	\item<3-> Two possible explanations with different implications for the design of health insurance system:
	\begin{itemize}
	\item<4-> Moral hazard
	\begin{itemize}
		\item<4-> Regulation of the premium (Community rating) on the health insurance market decreases physical activity
	\end{itemize}
	\item<5-> Bounded willpower
		\begin{itemize}
			\item<5-> Regulation of the premium that motivates insurers to practice cream-skimming potentially increases physical activity
\end{itemize}
\end{itemize}
\end{itemize}
\end{frame}
\subsection{RQ}
\begin{frame}{Research Question}
	Does community rating with cream-skimming induce more physical activity than a risk-rating health insurance setting?
\end{frame}
\section{Theoretical framework}
\subsection{Insured's decision about the level of prevention}
\begin{frame}{Privately optimal level of prevention}
	\begin{itemize} 
			\item Risk rating contract
			$$c'(p)=\delta(\Delta{H}+\Delta\gamma)$$  
			\item Community rating contract
			$$c'(p)=\delta(\Delta{H})$$
			\item Cream-skimming contract
			$$c'(p)=\delta\frac{(\Delta{H})}{(1-b)}$$
	\end{itemize}
\end{frame}
\section{Experimental design}
\subsection{Basic information}
\begin{frame}{Health insurance game}
	\begin{itemize} 
		\item Subjects are divided into groups that imitate the incentives under the various insurance contracts 
		\begin{itemize} 
			\item RR treatment 
			\item CRCS treatment
		\end{itemize}
		\item Activity is tracked by fitness wristbands  
		\item The experiment lasts for three months
	\end{itemize}
\end{frame}
\subsection{Hypothesis}
\begin{frame}{Hypothesis}
  	\begin{itemize} 
  		\item \textbf{Hypothesis $0$:} Subjects in the RR treatment experienced a higher level of physical activity than subjects in CRCS during the monitored period.
  		\item\textbf{Hypothesis $1$:} Subjects in the RR treatment experienced a lower level of physical activity than subjects in CRCS during the monitored period.
  	\end{itemize}
  \end{frame}
  \section{Questions}
  \begin{frame}{Questions?}
  	\begin{figure}
\centering
\includegraphics[width=0.7\linewidth]{exercise-illustration}
\caption{}
\label{fig:exercise-illustration}
\end{figure}
  \end{frame}
\end{document}
